% \documentclass[sigconf,nonacm]{acmart}
\documentclass[manuscript,screen,nonacm]{acmart}


\usepackage{enumitem}
\setlist[enumerate,1]{label=\arabic*.}
\usepackage{siunitx}
\usepackage{hyperref}  % optional, for clickable links
\usepackage{cleveref}  % always load *after* hyperref
\usepackage{graphicx}
\usepackage{tabularx}
\usepackage{float}
\usepackage[table]{xcolor}
\usepackage{tikz}

\newcommand{\mypara}[1]{\smallskip\noindent\textbf{#1.}}

\AtBeginDocument{%
  \providecommand\BibTeX{{%
    Bib\TeX}}}

\settopmatter{printfolios=true}

\begin{document}

\title{Resilient Decentralized Communication for Grassroots Social Networks using BLE}

% \author{Dan Bachar}
% \affiliation{
%   \institution{Technische Universit{\"a}t M{\"u}nchen \\ TUM Chair for Connected Mobility}
%   \city{Munich}
%   \country{Germany}
% }
% \email{dan.bachar@tum.de}
\maketitle

\mypara{Setup}
We assume people with smartphones, each equipped with a keypair, unique whp. Public keys are public.
We assume a social graph (SG), in which people know the public key of their friends and the friends of their friends.
Furthermore, we assume that people use the last 128 bits of their public key as their Service UUID.  Thus, a person can know the Service UUID of their friends and their friends of friends.  Furthermore, we assume that smarphones aletrnate between advetising and scanning.

We are interested in two scenarios:
\begin{enumerate}
    \item \textbf{Given SG:} Can people find their friends and friends of friends and establish connection, for the purpose of direct communication as well as forwading?
    \item \textbf{Growing SG:} Can a person perform a ``cold call'' and introduce themselves to another person that is advertising, and establish a connection?
\end{enumerate}


\section{Goal}
\begin{enumerate}
  \item Enable communication in first-contact between complete strangers $U,V$ (no common friends of degree $\le n$), pending mutual consent
  \item Enable communication in first-contact between $n$-degree friends $U,V$: $U$ has common friends with $V$, or $U$ has friends who have common friends with friends of $V$ (someone can introduce)
  \item Enable communication after first-contact between peers of friendship degree $\le n$ in physical proximity of each other
    \item Given Social Graph (link topology) and a distance matrix, what is the most efficient way peers can communicate?
\end{enumerate}

\section{Problem}
\label{problem}
\begin{enumerate}
  \item What are the reliable delivery characteristics over BLE
  \item How to build a resilient ephemeral overlay network topology by establishing links between BLE devices
  \item \label{problem:privacy} Guarantee privacy and security of flood-based communication over BLE: direct clash with easier communication due to fingerprinting
  \item Decipher who can relay messages for whom: how to define friends of friends? what is our $n$?
\end{enumerate}

\section{Assumptions}
\begin{enumerate}
  \item All users have either Android or iPhone with BLE always enabled
  \item $0 \le n \le 2$: we consider stragers, friends, and friends-of-friends
  \item Users have an app installed that manages social graph using BLE communication
  \item Users consent to share public keys and (hashed) friend lists over BLE
  \item Users consent to store, carry, and forward messages for friends and friends-of-friends
\end{enumerate}

\section{Design}
\begin{enumerate}
  \item Use BLE advertisements to advertise presence and willingness to communicate and relay for others: advertise PK (+handle) (+hashed friend list?)
  \item GATT write/read (with response) to exchange messages (with delivery guarantees)
  \item Pairing/trust request sent over BLE to establish trust (friendship)
  \item Use a custom overlay on top of GATT to securely exchange messages using e2e encryption and bidirectional authentication
  \item Implement a relay mechanism to forward messages through mutual friends
\end{enumerate}

\section{Approach}
To address~\ref{problem}.~\ref{problem:privacy}, users can decide their openness to being befriended by others, and choose one of four privacy levels. Each level contains the previous level and adds a unique feature:
\begin{enumerate}
    \item Do not share anything: user does not participate in the relay of SG blocks
    \item Share PK: can relay SG blocks
    \item Share name: can be befriended by strangers in close proximity
    \item Share friends list: can be introduced to others not in close proximity
\end{enumerate}
$P$ and $Q$ can become friends in one of the three ways:
\begin{enumerate}
    \item Virtual handshakes in close proximity: $P$ and $Q$ use BLE advertisements to advertise their openness to making new friends; the advertisements contain their PKs, which they use as the primary correspondence address. When $P$ and $Q$ are sufficiently close to each other (detected through a sufficiently high RSSI), let them offer friendship to each other.
    \item Introduction through a mutual friend: a mutual friend of $P$ and $Q$, $R$ can introduce them by sharing their registered PKs. We assume $P$ and $Q$ know in-band addresses of one another, and request the introduction from $R$.
    \item Cold calling: $P$ knows an out-of-band address of $Q$, and sends $Q$ a Grassroots friendship request there, with $R$'s PK.
\end{enumerate}

\end{document}
